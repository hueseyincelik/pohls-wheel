% Document For Use With LuaLaTeX
\documentclass[ngerman,a4paper,11pt]{article}

% Encoding
\usepackage[ngerman]{babel}
\usepackage{fontspec,microtype}

% Enhancements
\usepackage[all]{nowidow}
\usepackage{parskip,csquotes,xcolor}
\usepackage[labelfont=bf]{caption}

\usepackage[allpages]{continue}
\renewcommand*{\flagcont}{\bf Nächste Seite beachten! $\rightarrow$}
\renewcommand*{\flagend}{}

% Math and Science
\usepackage[locale=DE,per-mode=fraction,separate-uncertainty,sticky-per]{siunitx}
\usepackage[tbtags]{mathtools}
\usepackage{amsfonts,amssymb,physics,interval}

% Hyperref
\usepackage[colorlinks,linkcolor={red!50!black},citecolor={blue!50!black},urlcolor={blue!80!black}]{hyperref}

% Load After Hyperref
\usepackage[margin=1in,includefoot]{geometry}
\usepackage[nameinlink,noabbrev]{cleveref}
\usepackage{float,xurl}

\title{\bf Themenkreis 7: Pohlsches Rad}
\author{\sc Hüseyin Çelik}
\date{WiSe 2022/23}

\begin{document}
\maketitle
\section*{Allgemeine Hinweise}
\begin{itemize}
  \item Die Auslenkung von Erreger und Schwinger wird mit einem Winkelgeber ausgelesen. Diese sind nicht ganz linear und müssen mit einem Polynom 6. Grades kalibriert werden;
  \item Die Resonanzfrequenz \textbf{nie} ohne Dämpfung einstellen! Bei den Pohlschen Rädern liegt ein Zettel bei, welchen Wert diese für das jeweilige Gerät annimmt;
  \item Der Maximalstrom für die Spulen beträgt \SI{2}{\ampere}. Mit den verwendeten (kleinen) Netzteilen sind aber nur \SI{1.7}{\ampere} erreichbar und es ist nicht möglich, die Spulen zu zerstören. Für den aperiodischen Grenzfall wird ein anderes Netzteil zusammen mit dem Tutor verwendet;
  \item Bei Wechsel der Erregerfrequenz den Einschwingvorgang beachten.
\end{itemize}
\section*{Software}
Die Software zur Ansteuerung des Arduinos und Messung des Erregers und Oszillators wurde in \textsc{python} vollständig neu geschrieben.\footnote{Fehlermeldungen, Anmerkungen, Verbesserungsvorschläge etc. zur Software können entweder auf GitHub unter \href{https://www.github.com/hueseyincelik/pohls-wheel}{\nolinkurl{github.com/hueseyincelik/pohls-wheel}} oder per E-Mail unter \href{mailto:celik@tu-berlin.de}{\nolinkurl{celik@tu-berlin.de}} eingereicht werden.}

Das Programm wird über die Schreibtischverknüpfung \enquote{TK7} gestartet. Das Konsolenfenster, welches sich im Hintergrund öffnet, sollte nicht geschlossen werden, kann aber ignoriert bzw. minimiert werden. Nach dem Programmstart ist in der Benutzeroberfläche nur der \enquote{CONNECT} Knopf sowie die Eingabe zum Serial Port zu sehen. Der standardmäßig eingestellte Port von \texttt{/dev/ttyACM0} sollte bei ordnungsgemäßer USB-Verbindung zu einer erfolgreichen Verbindung führen. Falls die Verbindung dennoch fehlschlägt, können die Ports \texttt{/dev/ttyACM1}, \texttt{/dev/ttyACM2} usw. durchprobiert werden. Schlägt die Verbindung weiterhin fehl, kann über die Konsole mithilfe des Befehls \texttt{ls /dev/tty*} eine Liste mit allen verfügbaren Ports ausgegeben werden, unter der sich auch der Arduino befinden sollte.

Nach erfolgreicher Verbindung mit dem Arduino werden in der Benutzeroberfläche die Knöpfe \enquote{START}, \enquote{STOP} und \enquote{SAVE} sowie die Eingabe für den Dateinamen und die Erregerfrequenz freigegeben (siehe \cref{fig:gui}).

Bei dem Dateinamen sollte beachtet werden, dass der Pfad relativ zum Homeverzeichnis des aktuellen Benutzers \enquote{student} angegeben wird. Soll die Datei auf dem Schreibtisch in einem Unterordner gespeichert werden, so muss dieser zuerst angelegt, und anschließend der Dateipfad in der Benutzeroberfläche ergänzt werden. \textbf{Vorhandene Dateien werden ohne zu fragen überschrieben!}

Die Messung kann mit der aktuell eingestellten Erregerfrequenz über den \enquote{START} Knopf gestartet und den \enquote{STOP} Knopf beendet werden. Wird bei laufender Messung der \enquote{START} Knopf gedrückt, so wird die Messung zunächst beendet, und bei erneutem Drücken neu gestartet. Wird bei laufender Messung der \enquote{SAVE} Knopf gedrückt, so wird die Messung beendet und die Messung mit dem angegebenen Dateinamen abgespeichert.
\begin{figure}[H]
  \centering
  \includegraphics[width=0.8\textwidth]{screenshot-gui.png}
  \caption{GUI zur Ansteuerung des Pohlschen Rades über den Arduino sowie zur Messung der Erreger- und Oszillatorfrequenz}
  \label{fig:gui}
\end{figure}
Gelegentlich bricht die Messung nach kurzer Zeit mit der Fehlermeldung

\hspace{1em} \enquote{Error: not enough values to unpack (expected 3)! Try restarting the Arduino!}

ab. Hier muss das Programm geschlossen, und der Arduino über das Ein- und Ausstecken der Stromzufuhr neu gestartet werden.
\section*{7.1 Gedämpfte Schwingung (1/2)}
Der Versuch ist zunächst wie im Buch durchzuführen. Werte, die für die Dämpfung beispielsweise eingestellt werden können, sind: \SI{0.5}{\ampere}, \SI{0.75}{\ampere}, \SI{1}{\ampere}, \SI{1.5}{\ampere}.
\subsection*{Aperiodischer Grenzfall \textemdash{} Nur zusammen mit Tutor durchführen}
Der aperiodische Grenzfall tritt bei den Rädern bei Strömen im Bereich von ca. \SIrange{2}{2.5}{\ampere} auf. Da die Spulen diesen Strom aber nicht dauerhaft aushalten, ist dieser Teilversuch zügig und zusammen mit dem Tutor durchzuführen. Dazu wird ein stärkeres Netzteil mit der Aufschrift: \enquote{Nur für aperiodischer Grenzfall bestimmt} verwendet. Dieses ähnelt den kleinen Netzteilen, verfügt aber über einen separaten Schalter zum aktivieren der Ausgangsspannung.
\end{document}
